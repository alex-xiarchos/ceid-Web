\chapter{ΥΛΟΠΟΙΗΣΗ}
    Η εφαρμογή μας σχεδιάστηκε και υλοποιήθηκε κυρίως με χρήση vanilla τεχνικών, όπως \textbf{PHP} και \textbf{JavaScript}.
    Για τον χειρισμό των αιτημάτων προς τον διακομιστή, χρησιμοποιήθηκε η βιβλιοθήκη \textbf{jQuery}, η οποία διευκολύνει την αποστολή αιτημάτων στην PHP μέσω της μεθόδου AJAX.

    Επιπλέον, για την κατασκευή των χαρτών στο frontend, αξιοποιήθηκε η βιβλιοθήκη \textbf{Leaflet}, που παρέχει διαδραστικούς και ευέλικτους χάρτες.
    Τέλος, για το responsive design της εφαρμογής χρησιμοποιήσαμε \textbf{Tailwind CSS} με Flexbox ώστε να έχουμε σωστή στοιχειοθέτηση σε πληθώρα αναλύσεων και συσκευών.

\subsection{ΧΑΡΤΗΣ ΥΛΟΠΟΙΗΣΗΣ}
    \begin{tblr}{
        colspec={>{\centering\arraybackslash}m{35mm}>{\centering\arraybackslash}m{35mm}>{\centering\arraybackslash}m{70mm}},
        row{2,4,6,8}={bg=lightgray!50}, row{3,5,7,9}={bg=lightgray}, row{1}={bg=black!90,fg=white}}
        Σελίδα                             & Javascript                & PHP                  \\
        \textbf{\texttt{index.html}}       & \texttt{login.js}         & \texttt{login.php} \\
        \textbf{\texttt{signup.html}}      & \texttt{signup.js}        & \texttt{register\_user.php} \\
        \textbf{\texttt{signup.html}}      & \texttt{signup.js}        & \texttt{register\_user.php} \\
    \end{tblr}